\chapter*{Conclusion}
Ce présent travail est réalisé dans le cadre du projet d'étude de cas décisionnelle. Il décrit l'ensemble des méthodes que nous avons employés pour développer un modèle de classification des objets astronomiques répondant à la problématique du challenge.
\newline Au cours de ce travail, nous avons appliqué les différents enchaînements du cycle de vie d'un projet datamining en nous basant sur la méthodologie de gestion de projet CRISP-DM comme méthode de développement.
\newline
Pour atteindre les objectifs que nous nous sommes fixés au début de ce projet, tout en nous conformant à la méthode de gestion de projet CRISP-DM, nous avons commencé par étudier la problématique du sujet, suivie de l'étape de la compréhension du problème et une étude descriptive des de données mises à notre disposition.
\newline Ensuite, nous avons enchaîné par l'étape de préparation des données, suivie de l'élaboration d'un modèle de classification répondant aux objectifs de la problématique.
\newline Ce travail nous a permis de tester, d'appliquer et d'améliorer nos connaissances théoriques et pratiques acquises durant nos cursus universitaires dans le domaine de datamining. De même, ce projet nous a offert la possibilité de nous confronter à la gestion d'un projet de datamining dans un milieu réel et d'avoir un début d'expérience sur certaines librairies et technologies utilisées.
